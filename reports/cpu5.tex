\documentclass{article}
\usepackage[utf8]{inputenc}
\usepackage{listings}
\usepackage[margin=.75in]{geometry}  % Sets the margins

\usepackage{color}
 
\definecolor{codegreen}{rgb}{0,0.6,0}
\definecolor{codegray}{rgb}{0.5,0.5,0.5}
\definecolor{codepurple}{rgb}{0.58,0,0.82}
\definecolor{backcolour}{rgb}{0.95,0.95,0.92}
 
\lstdefinestyle{mystyle}{
    backgroundcolor=\color{backcolour},   
    commentstyle=\color{codegreen},
    keywordstyle=\color{magenta},
    numberstyle=\tiny\color{codegray},
    stringstyle=\color{codepurple},
    basicstyle=\footnotesize,
    breakatwhitespace=false,         
    breaklines=true,                 
    captionpos=b,                    
    keepspaces=true,                 
    numbers=left,                    
    numbersep=5pt,                  
    showspaces=false,                
    showstringspaces=false,
    showtabs=false,                  
    tabsize=2
}
 
\lstset{style=mystyle}

\title{5200 CPU 4}
\author{Matthew Cather}
\date{March 2019}

\begin{document}

\maketitle

This section of the CPU project focuses on interfacing with our external memory module. To facilitate verifying the integration five additional ports were exposed on the top level module: \texttt{PC, WriteAddress, WriteData, ReadAddress, ReadAddress}. This allowed us to make sure that the correct data was entering and exiting the CPU during the program execution. Relying on the verification of the internal execution from the last project we can be reasonable sure that the CPU is correctly interfacing with the External Memory Module.

\tableofcontents
\clearpage

\section{External Memory unit}
\lstinputlisting[language=verilog]{components/Memory/Memory.sv}
\subsection{Test-bench}
\lstinputlisting[language=python]{components/Memory/Memory_test.py}

\clearpage
\section{CPU TOP}
\lstinputlisting[language=verilog]{components/CPU_TOP/CPU_TOP.sv}
\subsection{Test-bench}
\lstinputlisting[language=python]{components/CPU_TOP/CPU_TOP_test.py}

\clearpage
\section{Test Program}
The following program was used to verify the ability for the cpu to interface with the external memory. It used a variety of instructions to load an array from memory and add one to each value in the array and then write it back. To load the program the custom assembler was used along with the verilog directive \texttt{\$readmemb}.
\begin{lstlisting}
000 ADDI r1, r0, 7  # base for array address
001 ADDI r2, r0, r0 # index
002 ADDI r3, r0, 6  # size of array
003 ADDI r7, r0, -6 # jump offset
004 LUI  r1, r1, 1  # array@263
005 BEQ  r2, r3, 8
006 ADD  r4, r1, r2 # calculate offset address
007 LOAD r5, r4, 0  # load value from memory
008 ADDI r5, r5, 1  # Add one
009 ADD  r4, r4, r3 # calculate store address
010 STOR r5, r4, 0	# store value back in memory
011 ADDI r2, r2, 1  # ++index
012 JAL  r0, r7
013 BEQ  r0, r0, -1 # effectively halt
...
...
...
263 6
264 7
265 8
266 9
267 10
268 11
\end{lstlisting}

\subsection{Testbench Results}
\lstinputlisting{final_test.txt}

% \section{Assembler}
% \lstinputlisting[language=python]{assembler.py}
% \section{ISA.yaml}
% \lstinputlisting{isa.yaml}

\end{document}
