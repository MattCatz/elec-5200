\documentclass{article}
\usepackage[utf8]{inputenc}
\usepackage{listings}

\usepackage{color}
 
\definecolor{codegreen}{rgb}{0,0.6,0}
\definecolor{codegray}{rgb}{0.5,0.5,0.5}
\definecolor{codepurple}{rgb}{0.58,0,0.82}
\definecolor{backcolour}{rgb}{0.95,0.95,0.92}
 
\lstdefinestyle{mystyle}{
    backgroundcolor=\color{backcolour},   
    commentstyle=\color{codegreen},
    keywordstyle=\color{magenta},
    numberstyle=\tiny\color{codegray},
    stringstyle=\color{codepurple},
    basicstyle=\footnotesize,
    breakatwhitespace=false,         
    breaklines=true,                 
    captionpos=b,                    
    keepspaces=true,                 
    numbers=left,                    
    numbersep=5pt,                  
    showspaces=false,                
    showstringspaces=false,
    showtabs=false,                  
    tabsize=2
}
 
\lstset{style=mystyle}

\title{5200 CPU 3}
\author{Matthew Cather}
\date{March 2019}

\begin{document}

\maketitle

For this section of the project I chose to model the CPU using the python library \texttt{py-MTL}. I chose this it allows me to focus and implement more complex ideas while still remaining productive. Ultimately I would like to transform this into a pipe-lined processor once the single cycle design is verified. 

\tableofcontents
\clearpage

\section{Registers}
\subsection{Immediate Register}

\subsection{Program Counter Register}

\section{Multiplexers}
Utilizing the paramiterzation features of py-MTL I am able to use one generic Multiplexer model for all multiplexers required by the design:
\begin{itemize}
    \item Data Mux
    \item Operand Mux
    \item PC Mux
\end{itemize}
\lstinputlisting[language=python]{components/MUX/MUX.py}

\section{Register File}
\lstinputlisting[language=python]{components/RegFile/RegFile.py}

\section{Arithmetic and Logic unit}
\lstinputlisting[language=python]{components/ALU/ALU.py}

\section{PC Adder}

\end{document}
